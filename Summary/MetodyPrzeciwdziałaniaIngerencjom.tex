\chapter{Metody przeciwdziałania ingerencjom sieciowym} \label{rozdz.metody.przeciwdzialania} 

Prężny rozwój technologii związanych z bezpieczeństwem jest w dużym stopniu uwarunkowany pojawianiem się coraz doskonalszych technik penetracji i włamań. Wiele korporacji i społeczności podejmuje się wyzwania stworzenia odpowiedniej metodyki zwalczania takich zachowań. Przez lata postępu dziedziny bezpieczeństwa powstało wiele podejść. Dość często w celu zapewnienia skutecznej ochrony potrzebne jest zestawienie różnorodnych rozwiązań.

Do nieco starszych struktur należą systemy wykrywania ingerencji IDS (ang. Intrusion Detection Systems) służące do monitorowania sieci w celu wykrycia złośliwych działań. Informacje uzyskane w ten sposób gromadzone są w bazie danych, bądź dziennikach zdarzeń. W wyniku ustalenia podejrzanego zachowania składany jest zazwyczaj raport stronie administracyjnej. Wspomniane oprogramowanie nie posiada mechanizmu zatrzymywania nieautoryzowanych prób dostępu do zasobów. Z tego powodu często zestawia się je z innymi technologiami. Może pracować również jako sniffer przechwytujący pakiety wędrujące przez media komunikacyjne i protokoły, zazwyczaj TCP/IP.

Aktualnie dużo szybszy rozwój odnotowują rozwiązania typu IPS (ang. Intrusion Prevention System) podobne w swojej mechanice do systemów IDS z nieco szerszym zastosowaniem (różnice przedstawiono na rysunku 1). Od narzędzi tej klasy wymagane jest, by prócz wykrywania ataków, skuteczne je blokowały. Działają na podobnej zasadzie co systemy firewall. Ruch wchodzi do urządzenia poprzez interfejs sieciowy i poddawany jest analizie, po czym opuszcza go innym interfejsem. Technologia IPS zapewnia aktywne przeciwstawianie się atakom pracując w trybie in-line. Tryb ten pozwala na blokowanie ingerencji w czasie rzeczywistym, broniąc zasobów do których próbuje dostać się włamywacz. Wpływa to na łatwiejszą kontrolę zdarzeń, unikanie fałszywych alarmów, zmniejszenie ryzyka zgubienia pakietów. 

\vspace{10mm} \hspace{-13mm} \noindent
\includegraphics[scale=0.7]{IDSvsIPS.png}
\figurename{1. Schemat działania sieciowego IPS oraz IDS ze zintegrowanym firewall'em. (przykładowy obraz, zostanie poprawiony)}

Firewall jest zbiorem powiązanych programów usytuowanych obok interfejsu sieciowego serwera chroniących zasoby sieci prywatnej przed użytkownikami sieci publicznej. Mechanizm ten monitoruje przepływ ruchu pomiędzy sieciami. Pracując ściśle z ruterem filtruje pakiety w celu określenia, które z nich mogą zostać dostarczone do ich miejsca docelowego. Firewall'e często instalowane są poza cała siecią dlatego żadne zewnętrzne żądanie nie dostanie się bezpośrednio do zasobów prywatnych. Dzięki dostarczonemu modułowi ustawień, można dostosowywać je do polityki bezpieczeństwa danej organizacji. Wiele implementacji zapór posiada mechanizm wykrywania ingerencji. Ruch wejściowy i wyjściowy analizowany jest na podstawie zdefiniowanych reguł.

Ciekawe podejście zastosowane jest w narzędziach typu Honeypot. Stanowią one element wabiący oraz pułapkę na użytkowników, którzy próbują nielegalnie dostać się do zasobów systemu host'a. Z założenia wszelkie wykryte interakcje z tymi urządzeniami traktowane są jako złośliwe. Każde oprogramowanie honeypot jest unikalne i nie rozwiązuje konkretnego problemu. Są to jednak rozwiązania bardzo elastyczne i mogą zostać wykorzystane do wykrywania ataków, ich zapobiegania, bądź zbierania i analizy informacji. Dobrym przykładem zastosowania takiego podejścia może być ulokowany w sieci system operacyjny starszej generacji z kontem administracyjnym bez ustawionego hasła.

Z tej krótkiej charakterystyki najważniejszych osiągnięć w dziedzinie bezpieczeństwa wynika, że istnieje wiele podejść o kompleksowym zastosowaniu oraz techniki mniej wyspecjalizowane, stworzone do konkretnych zadań. Mogłoby sie wydawać, że skoro systemy IPS posiadają wszystkie niezbędne funkcje to nie warto stosować innych rozwiązań. Zapewne w wielu przypadkach takie podejście ma swoje uzasadnienie. Dobrze zaimplementowany produkt tej architektury mógłby sprostać zapotrzebowaniu na elementy bezpieczeństwa wielu systemów. Istnieje jednak kilka cenionych implementacji podejścia IDS, które są dopiero na etapie opracowywania technik blokowania ingerencji. Z drugiej strony nowoczesne systemy operacyjne dostarczają w standardzie coraz bardziej innowacyjne firewall'e. Może zajść potrzeba doprowadzenia do konfrontacji systemów wykrywania ingerencji z nowoczesnymi zaporami sieciowymi. Realizacja takiej sytuacji opisana jest w niniejszej pracy magisterskiej.

Do rozwiązania przedstawionej problematyki zostało wykorzystane środowisko oparte o system operacyjny CentOS posiadający dwie implementacje zapór sieciowych, a mianowicie iptables i firewalld. Różnią się one głównie sposobem konfiguracji. Narzędzie firewalld w przeciwieństwie do swojego rywala umożliwia modyfikacje ustawień w czasie uruchomienia. Nie narzuca konieczności restartu firewall'a przy każdej dokonanej zmianie, co daje dużą dynamikę i płynność pracy. Zestawiając taką technologię z najlepszym obecnie darmowym system wykrywania ingerencji Snort, można uzyskać solidne zabezpieczenie. W analizowanym przypadku niezbędne jest dostarczenie nowego modułu komunikacyjnego dla tych narzędzi, który został przygotowany w ramach projektu. 

Na rynku oprogramowania dostepne jest narzędzie SnortSam o podobnym schemacie działania. Nie jest ono jednak rozwijane od dawna, dlatego nie obsługuje najnowszych rozwiązań w zakresie reguł filtrowania pakietów. Zarządza przede wszystkim starszymi zabezpieczeniami o statycznym działaniu. SnortSam nie jest nakładką Snorta, stanowi oddzielną implementację, przez co zużywa dodatkowe zasoby. Warto zatem podjąć się tej tematyki w celu sprawdzenia w jakim stopniu współpraca nowych osiągnięć technologicznych wpływa na bezpieczeństwo systemów w sieci.

Dzięki popularności zagadnienia bezpieczeństwa, dostępny jest szeroki dobór pozycji literaturowych. Są to zazwyczaj źródła elektroniczne twórców oprogramowania napisane w języku angielskim, choć nie brakuje polskich publikacji. Dokumentacje techniczne gotowych rozwiązań można znaleźć na oficjalnych stronach producentów, co daje nam sposobność uzyskania rzetelnych i aktualnych informacji. Wiedzę na ten temat można także czerpać z pozycji książkowych, instrukcji, artykułów naukowych, konferencji i wielu innych zasobów. 